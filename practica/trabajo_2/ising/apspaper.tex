\documentclass[%
 reprint,
 amsmath,amssymb,
 aps,
]{revtex4-1}

\usepackage{graphicx}
\usepackage{dcolumn}
\usepackage{bm}

\begin{document}

\preprint{APS/123-QED}

\title{Ising 2D}

\author{Felipe Gonzalez}

\affiliation{%
 Departamento de F\'\i sica, Facultad de Ciencias Exactas y Naturales, Universidad de Buenos Aires,\\
 Pabell\'on I, Ciudad Universitaria, 1428 Buenos Aires, Argentina.
}

\date{\today}

\begin{abstract}
Se estudio la evoluci\'on de un sistema de Ising ...


\begin{description}
\item[PACS numbers]
45.70.Vn, 89.65.Lm
\end{description}
\end{abstract}

\pacs{45.70.Vn, 89.65.Lm}

\maketitle

\section{\label{intro}Introducci\'on}

El modelo de Ising es un modelo f\'isico desarrollado para estudiar el
comportamiento de materiales ferromagnéticos, pero que es de inter\'es para el
estudio de fen\'omenos cr\'iticos en general. Es el modelo paradigm\'atico para
el estudio de transiciones de fase, presentando una de ellas cuando la
temperatura del sistema decrece a un valor cr\'itico tal que surge
magnetizaci\'on espont\'anea.

Este modelo fue inventado por el f\'isico Wilhelm Lenz, que lo concibió
como un problema para su alumno Ernst Ising para demostrar que el sistema
presentaba una transici\'on de fase. Ising demostró que en una dimensión
no existía tal transici\'on de fase, resolviéndolo en su tesis de 1924. El modelo
bidimensional de Ising de retícula cuadrada solo pudo ser descripto
analíticamente mucho más tarde, por Lars Onsager, que demostró que
efectivamente se encuentra una transici\'on de fase a un temperatura cr\'itica
$T_c > 0$\cite{Onsager}.

\section{\label{theory}El modelo}

El modelo de Ising se constituye por un arreglo bidimensional de espines
est\'aticos con orientaci\'on \textit{up} o \textit{down}, los cuales
evolucionan de manera "aleatoria" en funci\'on de la temperatura de la muestra
y la energ\'ia de interacci\'on entre espines. Esta evoluci\'on esta
caracterizada por un Metropolis Monte Carlo, el cual ser\'a explicado en la
proxima secci\'on.

\subsection{\label{transition}Metropolis Monte Carlo}

\subsection{\label{transition}Modelo de Ising}

El modelo de Ising estudiado consiste en una red bidimensional infinita,
rellena de espines con proyecci\'on \textit{up} o \textit{down}, caracterizada
por el siguiente Hamiltoniano.

\begin{equation}
  H = - \sum_i^N s_i B - J \sum_{<i, j>} s_i s_j,
\end{equation}

siendo, $s_i$ la proyecci\'on de cada esp\'in, $B$ un campo magn\'etico externo,
$J$ la energia de interacci\'on entre cada par de esp\'i y $<i, j>$ simboliza la
sumatoria que cuenta \'unicamente los primeros vecinos de cada esp\'in. Vemos que
no contamos con una energ\'a cin\'etica, por lo que cada esp\'in se encuentra
est\'atico.

Existen soluciones anal\'iticas para este problema. Para el caso de $J = 0, B
\neq 0$ tenemos que el problema se puede resolver plantenado un sistema
can\'onico, en el cual cada esp\'in no interactua por lo que se puede plantear
lo siguiente.

% Solucion para J = 0, buscar en carpeta de Teo 3.

Otro caso muy interesante es el caso de $B = 0, J \neq 0$. Para este caso,
Onsager\cite{Onsager} demuestra que existe $T_c = \frac{}{}$ tal que para $T <
T_c$ se tiene que ocurre magnetizaci\'on espontanea, que corresponde a una
transici\'on de fase de segundo orden (continua, con derivada discontinua) cuyo
par\'ametro de orden es la magnetizaci\'on media de la red.

Sobre este trabajo se utiliza la siguiente convenci\'on, en la cual se toma
$k_b = 1$, $J \rightarrow \frac{J}{T}$ y $B \rightarrow \frac{B}{T}$.


\section{\label{simulations}Simulaciones num\'ericas}

Para realizar simulaciones num\'ericas cuya validez se sostenga, se tuvo
especial cuidado con las siguientes cuestiones.

Se comenz\'o por realizar un estudio sobre longitud de correlaci\'on y
termalizacion de la red en funcion de los distintos valores de temperatura, en
pos de asegurar que las muestras tomadas sean representativas de la temperatura
en la cual se setea el sistema. Para esto se realizo un barrido para $J \in
[0.1, 0.6]$, en una red de lado $L = 32$ con $300.000$ pasos,
con $300.000 >> L^2 = 1024$, con la intenci\'on de estar completamente seguro
de que para cada $J$ se llega a un estaodo representativo de la temperatura y
que no corresponde a un transitorio, sino al estacionario. Sobre la figura ...
se puede ver la magnetizaci\'on media de la muestra en funcion de cada paso,
para $J < J_c$ y $J > J_c$. Vemos que para temperaturas bajas $(J > J_c)$
tenemos que la cantidad de pasos para termalizar el sistema se vuelve muy
grande superando ampliamente el estimado inicial de $L^2$, mientras que para
$J < J_c$ el sistema llega a un estado representativo de la temperatura en
relativamente pocos casos.

Otro fen\'omeno interesante surge al estudiar la correlaci\'on de las
mediciones. Vemos en la figura ... que para $J \simeq J_c$, la muestra muestra
una gran correlaci\'on para varios pasos, ya que se puede ver una especie de
"forma funcional" para puntos separados por muchos mas de $L^2$ pasos,
comparado con el caso de $J < J_c$. Comparando las longitudes de correlaci\'on
(figura ...) se vuelve muy claro como para $J_c$ es importante que el sampleo
de magnitudes ocurra cada una gran cantidad de pasos, en pos de asegurar
descorrelaci\'on entre mediciones.

\section{\label{results}Resultados}

\subsection{\label{p_c} Determinaci\'on de $p_c(\infty)$ y $\nu$, y $\tau$}

\section{\label{conclusions}Conclusiones}



\appendix
%\bibliography{apssamp}% Produces the bibliography via BibTeX.
\bibliography{apspaper}% Produces the bibliography via BibTeX.

\end{document}
